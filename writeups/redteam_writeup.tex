\documentclass[12pt]{article}
\usepackage{color}
\usepackage[pdftex]{graphicx}
\usepackage{tikz}
\usepackage{enumerate}
\usepackage[margin=1in,footskip=0.25in]{geometry}
\usepackage[normalem]{ulem}
\usepackage{setspace}
\usepackage{fancyvrb}
\begin{document}
\noindent
Brian Sheedy \& Avi Weinstock \hfill Cryptography \& Network Security (\Verb|CSCI-4230|)\\
Red Team Crypto Writeup for Attacks on Mock Banking System\\
Target Implementation Written By Peter Kang, Kevin Andrade, and Julius Alexander IV
\doublespace

\section*{Successful attacks}
\subsection*{Denial of Service (Bank Process Killed)}
\subsection*{Denial of Service (Permanent 100\% CPU Usage On Host)}
\subsection*{Man in the Middle Attack}
The way the initial handshake is supposed to work is:
\begin{enumerate}[1)]
\item Bank generates an RSA keypair
\item Bank sends ATM its public key
\item ATM generates random AES key \& IV
\item ATM encrypts the key \& IV with the Bank's public key
\item ATM sends the encrypted key \& IV to the Bank
\item Bank decrypts the key \& IV
\item All further conversation is fixed-width packets encrypted with AES
\end{enumerate}
(Un)fortunately, the ATM has no way of knowing that the RSA public key actually belongs to the bank, and hence all traffic can be intercepted and modified with the following protocol:
\begin{enumerate}[1)]
\item Bank generates an RSA keypair
\item Bank sends Proxy its public key
\item Proxy generates an RSA keypair
\item Proxy sends ATM its public key
\item ATM generates random AES key \& IV
\item ATM encrypts the key \& IV with the Proxy's public key
\item ATM sends the encrypted key \& IV to the Proxy
\item Proxy decrypts the key \& IV
\item Proxy re-encrypts the key \& IV with the Bank's public key
\item All further conversation is fixed-width packets encrypted with AES, and the proxy knows the key
\end{enumerate}
\section*{Attempted unsuccessful attacks}
\subsection*{Remote Code Execution}
\end{document}
