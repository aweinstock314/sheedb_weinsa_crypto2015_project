\documentclass[12pt]{article}
\usepackage{color}
\usepackage[pdftex]{graphicx}
\usepackage{tikz}
\usepackage{enumerate}
\usepackage[margin=1in,footskip=0.25in]{geometry}
\usepackage[normalem]{ulem}
\usepackage{setspace}
\usepackage{fancyvrb}
\begin{document}
\noindent
Brian Sheedy \& Avi Weinstock \hfill Cryptography \& Network Security (\Verb|CSCI-4230|)\\
Red Team Crypto Writeup for Attacks on Mock Banking System\\
Target Implementation Written By Peter Kang, Kevin Andrade, and Julius Alexander IV
\doublespace

\section*{Successful attacks}
\subsection*{Denial of Service (Bank Process Killed)}
\\There are two ways of denying service by killing the bank process. The first is by causing an SIGPIPE signal to be sent from a closed socket, which causes the bank process to close immediately. This can be achieved by rapidly creating a few ATMs, sending login/logout messages, and then immediately killing the processes. The script dos_connection.py achieves this with small (0.1 seconds or less, depending on processor speed) sleeps in between commands.
\\The second way of killing the bank process is to send back a malformed message back to the bank when it is expecting an RSA encrypted AES key that is 384 bytes long. If it receives fewer bytes than expected, CryptoPP throws an exception and kills the bank. This is easily achieved by modifying the proxy to forward an empty string when it receives the actual encrypted key. MitMProxyMalformedRSA.hs does this on line 101. Using this proxy, connect to the bank via an ATM and then kill the proxy to close the socket to the bank. Killing the proxy is necessary, as otherwise the bank will wait indefinitely for the remaining bytes it expects.
\subsection*{Denial of Service (Permanent 100\% CPU Usage On Host)}
\\There is a bug in the bank’s code that causes threads handling connections to ATMs to get stuck in an infinite loop whenever the connected ATM is closed (such as via ^C). Since each ATM connection is handled by a separate thread, repeating this process multiple times causes multiple threads to be stuck in infinite loops, causing the bank’s host to run even slower. This can be achieved by using dos_connection.py and setting the sleeps to 1 second each. The script will continue to create and kill ATM processes until the bank refuses any further connections. Since the threads will never end, the sockets for each ATM will also never close, causing the bank to refuse connections indefinitely once enough ATMs have connected and disconnected.
\subsection*{Man in the Middle Attack}
The way the initial handshake is supposed to work is:
\begin{enumerate}[1)]
\item Bank generates an RSA keypair
\item Bank sends ATM its public key
\item ATM generates random AES key \& IV
\item ATM encrypts the key \& IV with the Bank's public key
\item ATM sends the encrypted key \& IV to the Bank
\item Bank decrypts the key \& IV
\item All further conversation is fixed-width packets encrypted with AES
\end{enumerate}
(Un)fortunately, the ATM has no way of knowing that the RSA public key actually belongs to the bank, and hence all traffic can be intercepted and modified with the following protocol:
\begin{enumerate}[1)]
\item Bank generates an RSA keypair
\item Bank sends Proxy its public key
\item Proxy generates an RSA keypair
\item Proxy sends ATM its public key
\item ATM generates random AES key \& IV
\item ATM encrypts the key \& IV with the Proxy's public key
\item ATM sends the encrypted key \& IV to the Proxy
\item Proxy decrypts the key \& IV
\item Proxy re-encrypts the key \& IV with the Bank's public key
\item All further conversation is fixed-width packets encrypted with AES, and the proxy knows the key
\end{enumerate}
Since the proxy can read and modify all traffic (as well as start independent sessions without an ATM connected), if anyone tries to log in with the proxy in the middle the proxy can capture their PIN and transfer/withdraw all their money before logging them in.
\section*{Attempted unsuccessful attacks}
\subsection*{Remote Code Execution}
\end{document}
