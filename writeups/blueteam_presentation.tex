\documentclass{beamer}
\usepackage{tikz}
\title{BlueTeam presentation on Mock Banking System\\Cryptography \& Network Security (\Verb|CSCI-4230|)}
\date{December 9, 2015}
\author{Brian Sheedy \& Avi Weinstock}
\usepackage{fancyvrb}
\begin{document}
\maketitle

\begin{frame}[fragile]
\frametitle{Protocol Overview}
\begin{itemize}
\item Statically sized messages
\item AES Encryption
\item HMAC
\item Nonces
\end{itemize}
\end{frame}

\begin{frame}[fragile]
\frametitle{Encryption}
\begin{itemize}
\item 128-bit AES
\item Applied to everything except the sender's name and HMAC
\item Applied to every message, including nonce requests/responses
\item Provides confidentiality and authenticity
\item Keys statically set in bank, stored in cards for ATM
\end{itemize}
\end{frame}

\begin{frame}[fragile]
\frametitle{HMAC}
\begin{itemize}
\item OpenSSL's HMAC function using SHA-256
\item Applied to the encrypted data
\item Applied in every message
\item Provides integrity and authenticity
\item 128-bit key set statically in bank and stored in cards for ATM
\end{itemize}
\end{frame}

\begin{frame}[fragile]
\frametitle{Nonces}
\begin{itemize}
\item Prevent replay attacks
\item Each message to the bank preceeded by requesting a new nonce
\item Nonce included in the message
\item If the nonce from the ATM does not match the one in the bank, the bank rejects the message
\item If the nonce from the bank does not match the one originally sent to the bank, the ATM rejects the message
\item Nonce is cleared from the bank after every message
\end{itemize}
\end{frame}

\end{document}
